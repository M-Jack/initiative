\documentclass[12pt,letter,titlepage]{article}
% pour d\'efinir les caract\'eristiques g\'en\'erales du document
%!TEX TS-program = pdflatexmk
\usepackage[latin1]{inputenc}
\usepackage[T1]{fontenc}
\usepackage{amsfonts}
\usepackage{amsmath}
\usepackage{graphicx}
\usepackage{natbib}
\usepackage{listings}
\usepackage[toc,acronym]{glossaries}
\usepackage{hyperref}
\usepackage[usenames,dvipsnames]{xcolor}
\makeglossaries
\usepackage{fancyhdr}
\setlength{\headheight}{15.2pt}
\pagestyle{fancyplain}
\usepackage{pdfpages}
\usepackage{listings}
\usepackage{color}

\definecolor{dkgreen}{rgb}{0,0.6,0}
\definecolor{gray}{rgb}{0.5,0.5,0.5}
\definecolor{mauve}{rgb}{0.58,0,0.82}

\lstset{ %
  language=python,                % the language of the code
  basicstyle=\ttfamily,           % the size of the fonts that are used for the code
  numbers=left,                   % where to put the line-numbers
  numberstyle=\tiny\color{gray},  % the style that is used for the line-numbers
  stepnumber=2,                   % the step between two line-numbers. If it's 1, each line 
  % will be numbered
  numbersep=5pt,                  % how far the line-numbers are from the code
  backgroundcolor=\color{white},      % choose the background color. You must add \usepackage{color}
  showspaces=false,               % show spaces adding particular underscores
  showstringspaces=false,         % underline spaces within strings
  showtabs=false,                 % show tabs within strings adding particular underscores
  rulecolor=\color{black},        % if not set, the frame-color may be changed on line-breaks within not-black text (e.g. commens (green here))
  tabsize=2,                      % sets default tabsize to 2 spaces
  captionpos=b,                   % sets the caption-position to bottom
  breaklines=true,                % sets automatic line breaking
  breakatwhitespace=false,        % sets if automatic breaks should only happen at whitespace
  title=\lstname,                   % show the filename of files included with \lstinputlisting;
}

\newcommand{\specialcell}[2][c]{%
  \begin{tabular}[#1]{@{}c@{}}#2\end{tabular}}

\def\mytitle{CO250 :Assignment 1}
\title{\mytitle}
\definecolor{gray}{rgb}{0.3,0.3,0.3}
\hypersetup{linkbordercolor={1 1 1},
  colorlinks=true,
  linkcolor=gray}
\setcounter{tocdepth}{2}


\author{Bastien Jacot-Guillarmod}
\rhead{Christian Zommerfelds}
\chead{\today}
\lhead{Bastien Jacot-Guillarmod}


%%%%%%%%%%%%%%%%%%%%%%%%%%%%%%%%%%%%%%%%%%%%%%%%%%%%%%%%%%%%%%%%%%%%%%%%%%%%%%%%%%%%%%%%%%%%%%%%%%%%%%%%%%%%%%%%%%%%%%%%%%%%%%%
%%%%%%%%%%%%%%%%%%%%%%%%%%%%%%%%%%%%%%%%%%%%%%%%%%%%%%%%%%%%%%%%%%%%%%%%%%%%%%%%%%%%%%%%%%%%%%%%%%%%%%%%%%%%%%%%%%%%%%%%%%%%%%%

\begin{document}

\begin{center}
  {\bf \Large 
    ECE 456: Project Propasal\\
    OIS: Online Initiative System\\
    A Secure `We, the People'
  }
\end{center}
%%%%%%%%%%%%%%%%%%%%%%%%%%%%%%%%%%%%%%%%%%%%%%%%%%%%%%%%%%%%%%%%%%%%%%%%%%%%%%%%%%%%%%%%%%%%%%%%%%%%%%%%%%%%%%%%%%%%%%%%%%%%%%%
%%%%%%%%%%%%%%%%%%%%%%%%%%%%%%%%%%%%%%%%%%%%%%%%%%%%%%%%%%%%%%%%%%%%%%%%%%%%%%%%%%%%%%%%%%%%%%%%%%%%%%%%%%%%%%%%%%%%%%%%%%%%%%%

\section{Abstract}
We propose to implement a secure governmental website, to enable citizen of the country to sign popular initiatives in a secure way. It should provide protection mechanisms to ensure the same level of security as the pen and paper system.
\section{Problem statement}
`We, the People' is a petition system created by the White House that allows U.S. citizens to create and sign petitions online. Even if this is a great solution to increase the communication between taxpayers and the government, it is not suitable for initiatives. For instance, a non-U.S. citizen can easily create an account and sign petitions. Therefore, there is a need for a stricter protocol to ensure the indentity of a real voter.
\section{Our solution}
We propose to create a web server in a test environment which will provides the software prototype for such a system. A part of the registration process would not be electronic, therefore we will only plan this part and not implement it.\\
We will use a secure protocol such as HTTPS to establish an encrypted connection between the voters and the server.

\section{End-to-end security analysis}
\section{Pitfalls of using our approach that we didn't address}
\section{Usability issues}
\section{References}

 
\end{document}
